\chapter{最优化条件与对偶}
\section{最优化条件}
\subsection{无约束优化情形}
首先给出极小点的定义。
\begin{definition}[极小点]
    设$f:\mathbb{R}^n\to\bar{\mathbb{R}},X\subset\mathbb{R}^n$。若$\hat{x}\in X$满足条件:存在$\varepsilon >0$,使得$f(y)\geq f(\hat{x}),\forall y\in X$且$\|y-\hat{x}\|\leq \varepsilon$,则称$\hat{x}$为$f$在$X$上的局部极小点。若$f(y)\geq f(\hat{x})$对任意$y\in X$恒成立。则称$\hat{x}$为$f$在$X$上的全局极小点。
\end{definition}
\begin{theorem}
    设$f:\mathbb{R}^n\to\bar{\mathbb{R}}$在$\hat{x}$处可微,若$\hat{x}$是$f$在$\mathbb{R}^n$上的局部极小点,则有$\nabla f(\hat{x})=0$;若假设$f$二阶连续可微,则还有$\nabla^{2}f(\hat{x})\succeq 0$
\end{theorem}
\begin{proof}
    反设$\nabla f(\hat{x})\neq 0$,由$f$在$\hat{x}$处可微知
    \[
        f(y) = f(\hat{x}) + \langle \nabla f(\hat{x}),y-\hat{x} \rangle + o(\|y-\hat{x}\|)
    \]
    考虑$y(\tau) = \hat{x}-\tau\cdot \nabla f(\hat{x}),\tau>0$,带入上式可得
    \begin{equation}\label{eq:firstDerivate}
        f(y(\tau))=f(\hat{x})-\tau\cdot\|\nabla f(\hat{x})\|^2+o(\tau).
    \end{equation}
    由$\lim_{\tau\to 0_+}\frac{o(\tau)}{\tau} = 0$知存在$\tau_0>0$使得当$0<\tau<\tau_0$,
    \[
        \dfrac{o(\tau)}{\tau}<\dfrac{1}{2}\|\nabla f(\hat{x})\|^2
    \]
    于是由~(\ref{eq:firstDerivate})~知$ f(y(\tau))<f(\hat{x})-\frac{\tau}{2}\|\nabla f(\hat{x})\|^2<f(\hat{x}). $矛盾!

    若$f$在$\hat{x}$处二阶可微,则
    \begin{equation}\label{eq:secondDerivative}
        \begin{aligned}
            f(y)& =f(\hat{x})+\langle\nabla f(\hat{x}),y-\hat{x}\rangle+\frac12(y-\hat{x})^T\nabla^2f(\hat{x})(y-\hat{x})+o(\|y-\hat{x}\|^2)  \\
            &=f(\hat{x})+\frac12(y-\hat{x})^T\nabla^2f(\hat{x})(y-\hat{x})+o(\|y-\hat{x}\|^2).
        \end{aligned}
    \end{equation}
    反设$\nabla^2f(\hat{x})\prec 0$,固定非零向量$d\in\mathbb{R}^n$并考虑
    \[
        y(\tau)=\hat{x}+\tau d.
    \]
    将其代入~(\ref{eq:secondDerivative})~式可得
    \[
        f(y(\tau))=f(\hat{x})+\frac{\tau^{2}}{2}d^{T}\nabla^{2}f(\hat{x})d+o(\tau^{2}).
    \]
    类似的,存在$\tau>0$,使得
    \[
        o(\tau^2)<-\frac{\tau}{4}d^T\nabla^2f(\hat{x})d.
    \]
    从而
    \[
        f(y(\tau))<f(\hat{x})+\frac{\tau^{2}}{4}d^{T}\nabla^{2}f(\hat{x})d<f(\hat{x}).
    \]
    矛盾!
\end{proof}
\begin{theorem}
    设$f:\mathbb{R}^n\to \bar{\mathbb{R}}$为正规凸函数,以下结论成立
    \begin{enumerate}[label=(\roman*)]
        \item 若$\hat{x}$为$f$的局部极小,则$\hat{x}$也为$f$的全局极小点,且
        \begin{equation}\label{eq:zeroInPartial}
            0\in \partial f(\hat{x})
        \end{equation}
        \item 若~(\ref{eq:zeroInPartial})~成立,则$\hat{x}$为$f$的全局极小点
    \end{enumerate}
\end{theorem}
\begin{proof}
    \begin{enumerate}[label=(\roman*)]
        \item 设$\hat{x}$为$f$的局部极小点,对任意固定的$y\in\mathbb{R}^n$,必存在$1>\alpha_0>0$,使得当$0<\alpha<\alpha_0$时,
        \[
            \begin{array}{ll}
                f(\hat{x}) & \leq f(\hat{x} + \alpha(y-\hat{x}))\\
                & = f(\alpha y+(1-\alpha)\hat{x})\\
                & \leq \alpha f(y) + (1-\alpha)f(\hat{x})
            \end{array}
        \]
        从而$f(\hat{x})\leq f(y)$,可知$\hat{x}$为$f$的全局极小点,进而
        \begin{equation}\label{eq:fGeqHatx}
            f(y)\geq f(\hat{x}) + \left\langle 0,y-\hat{x} \right\rangle,\forall y \in\mathbb{R}^n
        \end{equation}
        因此,$0\in\partial f(\hat{x})$
        \item 当~(\ref{eq:fGeqHatx})~式成立,~(\ref{eq:zeroInPartial})~式必成立,从而$\hat{x}$为$f$的全局极小点。证毕!
    \end{enumerate}
\end{proof}