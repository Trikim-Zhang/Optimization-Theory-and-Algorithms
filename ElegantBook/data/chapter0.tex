\chapter{基础知识}
\begin{problemset}
    %------------------%
    % ------ 1 ------ %
    %------------------%
    \item 证明 Cauchy-Schwarz 不等式:
    \[
        \langle y,x \rangle \leqslant \| y \| \| x \|, \forall x,y\in \mathbb{R}^{n}
    \]
    \begin{proof}
        若$x=0$或$y = 0$,则结论显然成立。设$x \neq 0$,对任意数$\lambda$,取向量$y-\lambda x$,则有\cite{yangming2006}
        \[
            \langle y-\lambda x,y-\lambda x \rangle \geqslant 0
        \]
        即$\langle y,y \rangle + \lambda^2\langle x,x \rangle - 2\lambda \langle y,x \rangle \geqslant 0$
        由$x\neq 0$,可取到
        \[
            \lambda = \dfrac{\langle y,x \rangle}{\langle x,x \rangle}
        \]
        代入上式,即得
        \[
            \langle y,y \rangle\langle x,x \rangle\geqslant (\langle y,x \rangle)^2
        \]
        从而有
        \[
            \langle y,x \rangle \leqslant \| y \| \| x \|
        \]

        等式成立的充要条件为
        \[
            y-\lambda x = 0
        \]
        即$y$和$x$线性相关。证毕!
    \end{proof}

    \textbf{欧氏距离的三角不等式}:
    \begin{proof}
        \[
            \begin{array}{ll}
                \langle x+y,x+y \rangle &= \|x\|^2 + \|y\|^{2} + 2\langle x,y \rangle \\
                &\leqslant  \|x\|^2 + \|y\|^{2} + 2(\|x\|\|y\|)^2\quad \\
                &=(\|x\| + \|y\|)^2
            \end{array}
        \]
        所以,有$\| x+y \|\leqslant \|x\| + \|y\|$。
        
        等式成立的充要条件为$y$和$x$线性相关。证毕!
    \end{proof}
    \begin{enumerate}[label=(\roman*)]
        \item 欧氏范数$\ell_{1}-$范数$\|\cdot\|$的等价性:
        \[
            \frac{1}{\sqrt{n}}\|x\| \leqslant \|x\|_{\infty} \leqslant \|x\| \leqslant \|x\|_{1} \leqslant\sqrt{n}\cdot \|x\|_2, \ \forall x\in \mathbb{R}^{n}
        \]
        \begin{proof}
            \[
                \begin{array}{ll}
                    \dfrac{1}{\sqrt{n}}\|x\| &= \sqrt{\dfrac{\sum_{i = 1}^{n}x_i^2}{n}}  \leq \sqrt{\dfrac{n x_{\max}^2}{n}} = \|x\|_{\infty}\\
                    &\leq \sum\limits_{i = 1}^n\sqrt{|x|_i^2} = \|x\|\\
                    &\leq \sqrt{\sum\limits_{i = 1}^n|x|_i^2+2\sum\limits_{i\neq j}|x|_i|x|_j} = \sqrt{(|x_1|+\cdots+|x_n|)^2} = \|x\|_1\\
                    &=(1\cdot|x_1|+\cdots+1\cdot|x_n|)\leq \sqrt{n}\cdot\|x\|_2
                \end{array}
            \]
            证毕!
        \end{proof}
        \item 对任意的序列$\{a_{j}\}_{j=1}^{n}\subset\mathbb{R}$,均有:
        \[
            (\sum_{j=1}^na_j)^2+(\sum_{j=1}^n(-1)^ja_j)^2\leqslant(n+1)\sum_{j=1}^na_j^2.
        \]
        \begin{proof}
            $n = 2m$时,
            \[
                \begin{array}{ll}
                    &\left(\sum\limits_{i = 1}^{n}a_i\right)^2+\left(\sum\limits_{i = 1}^{n}(-1)^{j}a_i\right)^2 = \left(\sum\limits_{i = 1}^{m}a_{2i-1}+\sum\limits_{i = 1}^{m}a_{2i}\right)^2+\left(\sum\limits_{i = 1}^{m}a_{2i}-\sum\limits_{i = 1}^{m}a_{2i-1}\right)^2\\
                    &=2\left(\sum\limits_{i = 1}^{m}a_{2i-1}\right)^2+2\left(\sum\limits_{i = 1}^{m}a_{2i}\right)^2\text{\color{red}{由柯西不等式}}\\
                    &\leq 2\cdot m\sum\limits_{i = 1}^{m}a_{2i-1}^2+2\cdot m\sum\limits_{i = 1}^{m}a_{2i}^2\leq (2m+1)\sum\limits_{i = 1}^{2m}a_i^2 = (n+1)\sum\limits_{i = 1}^{n}a_i^2
                \end{array}
            \]
            $n = 2m+1$时,
            \[
                \begin{array}{ll}
                    &\left(\sum\limits_{i = 1}^{n}a_i\right)^2+\left(\sum\limits_{i = 1}^{n}(-1)^{j}a_i\right)^2 = \left(\sum\limits_{i = 1}^{m+1}a_{2i-1}+\sum\limits_{i = 1}^{m}a_{2i}\right)^2+\left(\sum\limits_{i = 1}^{m}a_{2i}-\sum\limits_{i = 1}^{m+1}a_{2i-1}\right)^2\\
                    &=2\left(\sum\limits_{i = 1}^{m+1}a_{2i-1}\right)^2+2\left(\sum\limits_{i = 1}^{m}a_{2i}\right)^2\text{\color{red}{由柯西不等式}}\\
                    &\leq 2\cdot (m+1)\sum\limits_{i = 1}^{m}a_{2i-1}^2+2\cdot m\sum\limits_{i = 1}^{m}a_{2i}^2\leq (2m+2)\sum\limits_{i = 1}^{2m+1}a_i^2 = (n+1)\sum\limits_{i = 1}^{n}a_i^2
                \end{array}
            \]
            综上所述,$(\sum_{j=1}^na_j)^2+(\sum_{j=1}^n(-1)^ja_j)^2\leqslant(n+1)\sum_{j=1}^na_j^2$。
            
            证毕!
        \end{proof}
    \end{enumerate}

    %------------------%
    % ------ 8 ------ %
    %------------------%
    \item 设$A\in \mathbb{R}^{n\times n}$为实对称矩阵,$b\in \mathbb{R}^{n}$。令
    \[
        f(x) = x^{\mathrm{T}}Ax-2b^{\mathrm{T}}x.
    \]
    证明下述结论等价:
    \begin{enumerate}[label=(\roman*)]
        \item $\inf\{f(x):x\in \mathbb{R}^n\}>-\infty$
        \item $A\succeq 0$且$b \in \operatorname{Im}A$
        \item $\operatorname{Arg}\min\{f(x):x\in\mathbb{R}^{n}\}\neq \emptyset$
    \end{enumerate}
    \begin{proof}
        $\color{red}{(i)\Rightarrow(ii)}$,首先证明$A\succeq 0$:
        
        假设$A\succeq 0$不成立,则$\exists v\in |\mathbb{R}^{n},\,\operatorname{s.t.} v^{\mathrm{T}}Av<0$,那么可以构建极小化序列$x = tv$使得
        % \[
        %     x^{\mathrm{T}}Ax = (tv)^{\mathrm{T}}A(t v)=t^{2}v^{\mathrm{T}}A v=t^{2}(v^{T}A v)< 0
        % \]

        % 那么
        \[
            \begin{array}{ll}
                f(x) &= t^2v^{\mathrm{T}}Av - 2tb^{\mathrm{T}}v\\
                     &= t(tv^{\mathrm{T}}Av - 2b^{\mathrm{T}}v)
            \end{array}
        \]

        那么$f(x)\rightarrow -\infty,\, t\rightarrow \infty$,与$\inf \{f(x):x\in\mathbb{R}^{n}\}> -\infty$矛盾。

        接下来证明$b\in \operatorname{Im}A$:

        假设$b\notin \operatorname{Im}A$,那么$b$可以写为$b = c+d$使得$c\in\operatorname{Im}A,d\in (\operatorname*{Im}A)^{\bot}=\operatorname*{Null}A^{\mathrm{T}}=\operatorname*{Null}A$

        那么可以构建极小化序列$x = kd$使得
        \[
            \begin{array}{ll}
                f(x) &= k^2d^{\mathrm{T}}Ad - 2kb^{\mathrm{T}}d\\
                &=- 2kc^{\mathrm{T}}d - 2kd^{\mathrm{T}}d\\
                &=- 2k\|d\|^2
            \end{array}
        \]
        则$f(x)\rightarrow -\infty,\, k\rightarrow \infty$,与$\inf \{f(x):x\in\mathbb{R}^{n}\}> -\infty$矛盾。
        % 假设$b\notin \operatorname{Im}A$,则矩阵$A$的列向量空间无法生成向量$b$

        % 那么根据列空间的定义,不存在一个向量$c$,使得$b = Ac$。现在令$x = c$,那么有
        % \[
        %     f(x) = x^{\mathrm{T}}Ax-2b^{\mathrm{T}}x = c^{\mathrm{T}}Ac-2b^{\mathrm{T}}c
        % \]
        % 由于$b\neq Ac$,所以可以选择一个$c$使得$2b^{\mathrm{T}}c>c^{\mathrm{T}}Ac$,使得$f(x)\rightarrow-\infty$。与$(i)$矛盾。

        $\color{red}{(ii)\Rightarrow (i)}$,存在正交阵$P$,$\operatorname{s.t.} A = P^{\mathrm{T}}DP$,其中$D = \operatorname{diag}\{\lambda_1,\cdots,\lambda_n\}$为对角阵,不妨设$\lambda_1\geq\lambda_2\geq\cdots\geq\lambda_n$,则
        \[
            f(x) = x^\mathrm{T}P^{\mathrm{T}}DPx-2b^{\mathrm{T}}P^{\mathrm{T}}Px
        \]
        令$y = Px$,则$f(x) = g(y) = y^{\mathrm{T}}Dy-2b^{\mathrm{T}}P^{\mathrm{T}}y$
        
        设$Pb = (b_1,\cdots,b_n)^{\mathrm{T}}$,则
        \[
            g(y) = \sum\limits_{i = 1}^{n}\lambda_iy_i^2-2\sum\limits_{i= 1}^{n}b_iy_i = \sum\limits_{i = 1}^{n}(\lambda_iy_i^2-2b_iy_i)
        \]

        因为$b\in\operatorname{Im}A$,存在$\alpha\in|\mathbb{R}^n$,$\operatorname{s.t.} Pb = PA\alpha = PP^{\mathrm{T}}DP\alpha = DP\alpha$。由$A\succeq 0$,有$\lambda_1\geq\lambda_2\geq\cdots\geq\lambda_n\geq 0$,那么当$\lambda_i = 0$时,$b_{i} = 0$,故而
        \[
            \inf\{f(x):x\in \mathbb{R}^n\}=\inf\{g(y):y\in \mathbb{R}^n\}>-\infty
        \]

        $\color{red}{(ii)\Rightarrow(iii)}$,进行$\color{red}{(ii)\Rightarrow(i)}$中的变量替换后
        \[
            \begin{array}{ll}
                &\operatorname{Arg}\min\{g(y):y\in\mathbb{R}^{n}\}\\
                {}&=\{y = [y_1,\cdots, y_n]:(y_i=\dfrac{b_i}{\lambda_i},\textbf{if}\lambda_i>0);(y_i\in |\mathbb{R}^n\textbf{if}\lambda_i=0)\}
            \end{array}
        \]
        故
        \[
            \operatorname{Arg}\min\{f(x):x\in \mathbb{R}^n\} = \left\{x = P^{\mathrm{T}}y:y\in\operatorname{Arg}\min\{g(y):y\in\mathbb{R}^{n}\}\right\}
        \]
        

        $\color{red}{(iii)\Rightarrow(i)}$,显然成立。
        
        证毕!
    \end{proof}
    %------------------%
    % ------ 10 ------ %
    %------------------%
    \item 设$A\in\mathbb{R}^{m\times n}$是一个秩为$r>0$的给定矩阵,$1\leq k\leq r$为给定的自然数.试求如下秩约束优化问题的解:
    \[
        \begin{array}{ll}
            \operatorname{minimize}&\|A-X\|_F\\
            \operatorname{subject\ to}&\operatorname{rank}X=k.
        \end{array}
    \]
    \begin{solution}
        $A = UDV^{\mathrm{T}}\Rightarrow U^{\mathrm{T}}AV = D$
        根据矩阵Frobenius范数的酉不变性
        \[
            \begin{array}{ll}
                \|A-X\|_F &= \|U^{\mathrm{T}}(A-X)V\|_F\\
                &=\|D-U^{\mathrm{T}}XV\|_F\\
            \end{array},
        \]
        令$Y = U^{\mathrm{T}}XV$上述优化问题可以转化为
        \[
            \begin{array}{ll}
                \operatorname{minimize}&\|D-Y\|_F\\
                \operatorname{subject\ to}&\operatorname{rank}Y=k.
            \end{array},
        \]
        其中$D = \operatorname{diag} (\sigma_1,\cdots,\sigma_r)$,$\sigma_1\geq\sigma_2\geq\cdots\geq\sigma_r$。
        易得,当$Y = \Sigma_k = \operatorname{diag} (\sigma_1,\cdots,\sigma_k,\cdots, 0)$时,达到最优解,此时
        \[
            X =  U\Sigma_kV^{\mathrm{T}}
        \]
    \end{solution}
  \end{problemset}